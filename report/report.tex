% Created 2012-12-15 Sat 21:00
\documentclass[11pt]{article}
\usepackage[utf8]{inputenc}
\usepackage[T1]{fontenc}
\usepackage{fixltx2e}
\usepackage{graphicx}
\usepackage{longtable}
\usepackage{float}
\usepackage{wrapfig}
\usepackage{soul}
\usepackage{textcomp}
\usepackage{marvosym}
\usepackage{wasysym}
\usepackage{latexsym}
\usepackage{amssymb}
\usepackage{hyperref}
\tolerance=1000
\providecommand{\alert}[1]{\textbf{#1}}

\title{Wstęp do algorytmów ewolucyjnych}
\author{Gleb Peregud}
\date{\today}
\hypersetup{
  pdfkeywords={},
  pdfsubject={},
  pdfcreator={Emacs Org-mode version 7.8.11}}

\begin{document}

\maketitle

\setcounter{tocdepth}{3}
\tableofcontents
\vspace*{1cm}

Dokumentacja końcowa

\section{Wstęp}
\label{sec-1}


Projekt polega na zaimplementowaniu benchmark CEC2005 w języku R,
użyciu pakietu DEoptim do zbadania efektywności optymizacji
Differential Evolution na tych funkcjach oraz porównaniu ich z
wynikami uzyskanymi przez J. Rönkkönen, S. Kukkonen oraz K. V. Price.

Dany projekt skupia się na zbadaniu następujących funkcji z benchmarku
CEC2005:
\begin{itemize}
\item Shifted Rosenbrock’s Function (funkcja numer 6)
\item Shifted Rotated Griewank’s Function without Bounds (funkcja numer 7)
\item Shifted Rotated Ackley’s Function with Global Optimum on Bounds (funkcja numer 8)
\item Shifted Rastrigin’s Function (funkcja numer 9)
\item Shifted Rotated Rastrigin’s Function (funkcja numer 10)
\item Shifted Rotated Weierstrass Function (funkcja numer 11)
\item Schwefel’s Problem 2.13 (funkcja numer 12)
\item Rotated Hybrid Composition Function with a Narrow Basin for the
  Global Optimum (funkcja numer 19)
\item Rotated Hybrid Composition Function with Global Optimum on the
  Bounds (funkcja numer 20)
\item Rotated Hybrid Composition Function (funkcja numer 21)
\item Rotated Hybrid Composition Function with High Condition Number
  Matrix (funkcja numer 22)
\item Non-Continuous Rotated Hybrid Composition Function (funkcja numer 23)
\item Rotated Hybrid Composition Function (funkcja numer 24)
\end{itemize}

Funkcje zostały zbadane dla 10 oraz 30 wymiarów.
\section{Algorytm}
\label{sec-2}


DEoptim jest implementacja algorytmu ewolucyjnej optymalizacji różniczkowej.

Parametry dla algorytmu optymalizacji zostały wybrane następujące:


\begin{center}
\begin{tabular}{rrrrr}
 Wymiary:  &   10  &   10  &   30  &   30  \\
\hline
  Funkcja  &   Np  &   Cr  &   Np  &   Cr  \\
\hline
        6  &   20  &  0.9  &   20  &  0.9  \\
        7  &   20  &  0.9  &   50  &  0.9  \\
        8  &   20  &  0.9  &  100  &  0.9  \\
        9  &   20  &  0.1  &   50  &  0.1  \\
       10  &  100  &  0.9  &   20  &  0.9  \\
       11  &   50  &  0.9  &   20  &  0.9  \\
       12  &  100  &  0.9  &   50  &  0.9  \\
       19  &  100  &  0.9  &  100  &  0.9  \\
       20  &  100  &  0.9  &  100  &  0.9  \\
       21  &  100  &  0.9  &  100  &  0.9  \\
       22  &  100  &  0.9  &  100  &  0.9  \\
       23  &  100  &  0.9  &  100  &  0.9  \\
       24  &  100  &  0.9  &  100  &  0.9  \\
\end{tabular}
\end{center}



To są takie same parametry jak były użyte przez Rönkkönen et
al. Analogicznie zastosowano strategie DE/local-to-best/1/bin.

Wyniki badanych funkcji są zbierane na etapach 10$^3$, 10$^4$, 10$^5$,
5*10$^5$ -tych wywołan funkcji optymalizowanej.

Optymalizacja dla każdej funkcja dla każdej ilości wymiarów została
uruchomiona 25 razy. Następnie zostały zestawione statystyki takie jak
0, 0.25, 0.5, 0.75 oraz 1 percentyl, średnia oraz standardowe
odchylenie wyników.
\section{Implementacja}
\label{sec-3}


Żeby nie powtarzać pracy wykonanej przez innych w projekcie został
wykorzystany projekt cec2005benchmark w języku R autorstwa Yasser
González Fernández <ygonzalezfernandez@gmail.com> oraz Marta Rosa Soto
Ortiz <mrosa@icimaf.cu>.

Implementacja użyta do uzyskania wyników jest dostępna na stronie
\href{https://github.com/gleber/cec2005bench.R}{https://github.com/gleber/cec2005bench.R}

Ważnym punktem implementacji była możliwość uruchamiania procesu
optymizacji po jej przerwaniu, żeby móc efektywnie wprowadzać jakieś
poprawki do systemu bez tracenia wyników jej pracy. Zostało to
zaimplementowane korzystając z możliwości serializacji macierzy i
zapisywaniu ich do pliku korzystając z funkcji 

Prawie cały ważny kod projektu się mieści w następujących linijkach:


\begin{verbatim}
 1:  for (i in funcs) { # for each of desired functions
 2:    limit = limits[i,] # fetch bounds for selected function
 3:    bias = fbias[i] # fetch bias for selected function
 4:    f = function(x) {
 5:      cec2005benchmark(i, x)
 6:    }
 7:    for (d in dims) { # for each of desired dimentions
 8:      lower = as.double(rep(limit[1], d)) # fetch upper bound for selected dimension
 9:      upper = as.double(rep(limit[2], d)) # fetch lower bound for selected dimension
10:      Np = params[i,sprintf('Np%dD', d)] # fetch Np for selected function and dimension
11:      Cr = params[i,sprintf('Cr%dD', d)] # fetch Cr for selected function and dimension
12:      for (run in 1:runs) { # run DE algorithm 'runs' times
13:        old.val = try(results.old[i, d, run, length(fes), 1]) # result caching mechanism
14:        if (is.numeric(old.val) && !is.na(old.val)) {
15:          results[i, d, run, , ] = results.old[i, d, run, , ]
16:          next;
17:        }
18:        r = DEoptim(f, lower, upper, control= # run optimization
19:          DEoptim.control(itermax=iters,
20:                          NP=Np, CR=Cr, F=F,
21:                          VTR=bias + eps,
22:                          strategy=strategy,
23:                          trace=iters / 10,
24:                          parallelType=1, packages=c('cec2005benchmark'), parVar=c('i') 
25:                          ))
26:        for (fe.i in 1:length(fes)) { # fetch error values from optimization history
27:          fe = fes[fe.i]
28:          fe.pop = min(r$optim$iter, (fe / fe.per.strategy[strategy]))
29:          pop = r$member$bestmemit[fe.pop,]
30:          val = f(pop)
31:          results[i, d, run, fe.i, 1] = val # saving results
32:          results[i, d, run, fe.i, 2] = r$optim$iter
33:          results[i, d, run, fe.i, 3] = r$optim$nfeval
34:        }
35:        save(results, file="results.rdata")
36:      }
37:    }
38:  }
\end{verbatim}

Główny kod wykorzystuje parametry zawarte w module `params.R`:


\begin{verbatim}
 1:  library('sfsmisc')
 2:  
 3:  funcs = c(6:11, 19:24)
 4:  fes = c(10**3, 10**4, 10**5, 5*10**5) # full workload
 5:  fes.max = last(fes)
 6:  dims = c(10, 30)
 7:  runs = 25
 8:  eps = 10**-8
 9:  strategy = 2 # strategy used by DEoptim
10:  fe.per.strategy = c(NA, 2)
11:  F = 0.9
12:  params = read.table('data/nps.tbl',header=T) # contains Np and Cr
13:  fbias = as.matrix(read.table('data/fbias_data.txt')) # contains minimum values
14:  limits = as.matrix(read.table('data/limits.txt')) # contains upper and lower limits for optimization domain
15:  
16:  iters = fes.max/fe.per.strategy[strategy] # defines number of iterations depending on the DE strategy used
\end{verbatim}

Powyższy kod uruchamia funkcje DEoptim z odpowiednimi parametrami z
włączonym trace-owaniem, żeby umożliwić zbieranie wyników ze
wszystkich interesujących nas iteracji. Jako, że funkcja DEoptim nie
obsługuje ograniczenia ilości iteracji na podstawie ilości wywołań
funkcji optymizowanej, została użyta następująca metoda zapewnienia
sobie takiej możliwości. W zależności od strategii optymizacji
różniczkowania ewolucyjnego w czasie każdej iteracji jest
wykonywana określona ilość wywołań funkcji optymizalizowanej. W naszym
przypadku, używając strategii DE/local-to-best/1/bin, przy każdej
iteracji funkcja optymalizacyjna jest wywoływana dwa razy.

Dzięki wyniesieniu wszystkich parametrów sterujących do oddzielnego
małego pliku, jest możliwość w łatwy sposób zarządzać nimi.

Dzięki użyciu biblioteki cec2005benchmark, która jest zaimplementowana
w C, implementacja funkcji optymalizowanych jest bardzo
szybka. Dodatkowo, w celu przyśpieszenia obliczeń została wykorzystana
możliwość uruchamiania procesu optymalizacji z wykorzystaniem
wszystkich procesorów dostępnych na sprzęcie (patrz linia 24).
\section{Wyniki}
\label{sec-4}


Tabelki wyników. Dla niektórych funkcji w polach wyników jest wpisane
NA, co oznacza, że wyniki jeszcze nie są dostępne, a obliczenia wciąż
trwają.

\texttt{
Results in 10 dimensions for functions 6, 7, 8, 9, 10, 11
% latex table generated in R 2.15.1 by xtable 1.7-0 package
% Sat Dec 15 19:01:56 2012
\begin{longtable}{llllllll}
  \hline
fe & stats & V6 & V7 & V8 & V9 & V10 & V11 \\ 
  \hline
1e+03 & 0\% & 1.2e-04 & 1.3e+03 & 2.0e+01 & 1.1e-08 & 2.0e+01 & 5.0e-01 \\ 
  1e+03 & 25\% & 1.4e-02 & 1.3e+03 & 2.0e+01 & 1.2e-08 & 2.4e+01 & 3.4e+00 \\ 
  1e+03 & 50\% & 1.5e-01 & 1.3e+03 & 2.1e+01 & 2.3e-08 & 2.8e+01 & 6.4e+00 \\ 
  1e+03 & 75\% & 2.2e+00 & 1.3e+03 & 2.1e+01 & 3.8e-08 & 3.0e+01 & 6.9e+00 \\ 
  1e+03 & 100\% & 4.1e+00 & 1.3e+03 & 2.1e+01 & 1.4e-05 & 3.7e+01 & 8.6e+00 \\ 
  1e+03 & avg & 1.2e+00 & 1.3e+03 & 2.1e+01 & 6.5e-07 & 2.7e+01 & 5.3e+00 \\ 
  1e+03 & std & 1.7e+00 & 1.5e-03 & 7.1e-02 & 2.8e-06 & 4.2e+00 & 2.7e+00 \\ 
  1e+04 & 0\% & 1.0e-08 & 1.3e+03 & 2.0e+01 & 1.1e-08 & 2.0e+00 & 1.0e-08 \\ 
  1e+04 & 25\% & 1.1e-08 & 1.3e+03 & 2.0e+01 & 1.1e-08 & 4.0e+00 & 1.1e-08 \\ 
  1e+04 & 50\% & 1.2e-08 & 1.3e+03 & 2.0e+01 & 1.5e-08 & 5.0e+00 & 4.1e-03 \\ 
  1e+04 & 75\% & 2.4e-08 & 1.3e+03 & 2.0e+01 & 2.3e-08 & 6.0e+00 & 1.5e+00 \\ 
  1e+04 & 100\% & 4.0e+00 & 1.3e+03 & 2.0e+01 & 3.9e-08 & 9.0e+00 & 2.6e+00 \\ 
  1e+04 & avg & 9.6e-01 & 1.3e+03 & 2.0e+01 & 1.8e-08 & 5.3e+00 & 6.9e-01 \\ 
  1e+04 & std & 1.7e+00 & 1.5e-13 & 7.1e-02 & 8.5e-09 & 1.7e+00 & 9.1e-01 \\ 
  1e+05 & 0\% & 1.0e-08 & 1.3e+03 & 2.0e+01 & 1.1e-08 & 2.0e+00 & 1.0e-08 \\ 
  1e+05 & 25\% & 1.1e-08 & 1.3e+03 & 2.0e+01 & 1.1e-08 & 4.0e+00 & 1.1e-08 \\ 
  1e+05 & 50\% & 1.2e-08 & 1.3e+03 & 2.0e+01 & 1.5e-08 & 5.0e+00 & 4.1e-03 \\ 
  1e+05 & 75\% & 2.4e-08 & 1.3e+03 & 2.0e+01 & 2.3e-08 & 6.0e+00 & 1.5e+00 \\ 
  1e+05 & 100\% & 4.0e+00 & 1.3e+03 & 2.0e+01 & 3.9e-08 & 8.0e+00 & 2.6e+00 \\ 
  1e+05 & avg & 9.6e-01 & 1.3e+03 & 2.0e+01 & 1.8e-08 & 5.1e+00 & 6.9e-01 \\ 
  1e+05 & std & 1.7e+00 & 0.0e+00 & 6.7e-02 & 8.5e-09 & 1.5e+00 & 9.1e-01 \\ 
  5e+05 & 0\% & 1.0e-08 & 1.3e+03 & 2.0e+01 & 1.1e-08 & 2.0e+00 & 1.0e-08 \\ 
  5e+05 & 25\% & 1.1e-08 & 1.3e+03 & 2.0e+01 & 1.1e-08 & 4.0e+00 & 1.1e-08 \\ 
  5e+05 & 50\% & 1.2e-08 & 1.3e+03 & 2.0e+01 & 1.5e-08 & 5.0e+00 & 4.1e-03 \\ 
  5e+05 & 75\% & 2.4e-08 & 1.3e+03 & 2.0e+01 & 2.3e-08 & 6.0e+00 & 1.5e+00 \\ 
  5e+05 & 100\% & 4.0e+00 & 1.3e+03 & 2.0e+01 & 3.9e-08 & 8.0e+00 & 2.6e+00 \\ 
  5e+05 & avg & 9.6e-01 & 1.3e+03 & 2.0e+01 & 1.8e-08 & 5.0e+00 & 6.9e-01 \\ 
  5e+05 & std & 1.7e+00 & 0.0e+00 & 4.6e-02 & 8.5e-09 & 1.4e+00 & 9.1e-01 \\ 
   \hline
\hline
\end{longtable}
\newpage
Results in 10 dimensions for functions 19, 20, 21, 22, 23, 24
% latex table generated in R 2.15.1 by xtable 1.7-0 package
% Sat Dec 15 19:01:56 2012
\begin{longtable}{llllllll}
  \hline
fe & stats & V19 & V20 & V21 & V22 & V23 & V24 \\ 
  \hline
1e+03 & 0\% & 3.0e+02 & 3.0e+02 & 3.0e+02 & NA & NA & NA \\ 
  1e+03 & 25\% & 8.0e+02 & 8.0e+02 & 3.0e+02 & NA & NA & NA \\ 
  1e+03 & 50\% & 8.0e+02 & 8.0e+02 & 3.0e+02 & NA & NA & NA \\ 
  1e+03 & 75\% & 8.0e+02 & 8.0e+02 & 5.0e+02 & NA & NA & NA \\ 
  1e+03 & 100\% & 8.0e+02 & 8.8e+02 & 9.0e+02 & NA & NA & NA \\ 
  1e+03 & avg & 6.8e+02 & 6.8e+02 & 4.1e+02 & NA & NA & NA \\ 
  1e+03 & std & 2.2e+02 & 2.2e+02 & 1.7e+02 & NA & NA & NA \\ 
  1e+04 & 0\% & 3.0e+02 & 3.0e+02 & 3.0e+02 & NA & NA & NA \\ 
  1e+04 & 25\% & 8.0e+02 & 8.0e+02 & 3.0e+02 & NA & NA & NA \\ 
  1e+04 & 50\% & 8.0e+02 & 8.0e+02 & 3.0e+02 & NA & NA & NA \\ 
  1e+04 & 75\% & 8.0e+02 & 8.0e+02 & 5.0e+02 & NA & NA & NA \\ 
  1e+04 & 100\% & 8.0e+02 & 8.8e+02 & 9.0e+02 & NA & NA & NA \\ 
  1e+04 & avg & 6.8e+02 & 6.8e+02 & 4.1e+02 & NA & NA & NA \\ 
  1e+04 & std & 2.2e+02 & 2.2e+02 & 1.7e+02 & NA & NA & NA \\ 
  1e+05 & 0\% & 3.0e+02 & 3.0e+02 & 3.0e+02 & NA & NA & NA \\ 
  1e+05 & 25\% & 8.0e+02 & 8.0e+02 & 3.0e+02 & NA & NA & NA \\ 
  1e+05 & 50\% & 8.0e+02 & 8.0e+02 & 3.0e+02 & NA & NA & NA \\ 
  1e+05 & 75\% & 8.0e+02 & 8.0e+02 & 5.0e+02 & NA & NA & NA \\ 
  1e+05 & 100\% & 8.0e+02 & 8.8e+02 & 9.0e+02 & NA & NA & NA \\ 
  1e+05 & avg & 6.8e+02 & 6.8e+02 & 4.1e+02 & NA & NA & NA \\ 
  1e+05 & std & 2.2e+02 & 2.2e+02 & 1.7e+02 & NA & NA & NA \\ 
  5e+05 & 0\% & 3.0e+02 & 3.0e+02 & 3.0e+02 & NA & NA & NA \\ 
  5e+05 & 25\% & 8.0e+02 & 8.0e+02 & 3.0e+02 & NA & NA & NA \\ 
  5e+05 & 50\% & 8.0e+02 & 8.0e+02 & 3.0e+02 & NA & NA & NA \\ 
  5e+05 & 75\% & 8.0e+02 & 8.0e+02 & 5.0e+02 & NA & NA & NA \\ 
  5e+05 & 100\% & 8.0e+02 & 8.8e+02 & 9.0e+02 & NA & NA & NA \\ 
  5e+05 & avg & 6.8e+02 & 6.8e+02 & 4.1e+02 & NA & NA & NA \\ 
  5e+05 & std & 2.2e+02 & 2.2e+02 & 1.7e+02 & NA & NA & NA \\ 
   \hline
\hline
\end{longtable}
\newpage
Results in 30 dimensions for functions 6, 7, 8, 9, 10, 11
% latex table generated in R 2.15.1 by xtable 1.7-0 package
% Sat Dec 15 19:01:56 2012
\begin{longtable}{llllllll}
  \hline
fe & stats & V6 & V7 & V8 & V9 & V10 & V11 \\ 
  \hline
1e+03 & 0\% & 2.7e+02 & 4.8e+03 & 2.1e+01 & 6.0e+00 & 1.8e+02 & 3.4e+01 \\ 
  1e+03 & 25\% & 5.2e+02 & 4.8e+03 & 2.1e+01 & 7.8e+00 & 2.2e+02 & 3.8e+01 \\ 
  1e+03 & 50\% & 6.0e+02 & 4.8e+03 & 2.1e+01 & 9.0e+00 & 2.4e+02 & 3.8e+01 \\ 
  1e+03 & 75\% & 8.7e+02 & 4.8e+03 & 2.1e+01 & 9.6e+00 & 2.5e+02 & 3.9e+01 \\ 
  1e+03 & 100\% & 1.5e+03 & 4.8e+03 & 2.1e+01 & 1.4e+01 & 2.7e+02 & 4.1e+01 \\ 
  1e+03 & avg & 7.1e+02 & 4.8e+03 & 2.1e+01 & 9.0e+00 & 2.3e+02 & 3.8e+01 \\ 
  1e+03 & std & 3.1e+02 & 1.5e+01 & 3.6e-02 & 1.9e+00 & 2.1e+01 & 1.5e+00 \\ 
  1e+04 & 0\% & 1.0e-08 & 4.7e+03 & 2.1e+01 & 1.0e-08 & 4.2e+01 & 1.4e+01 \\ 
  1e+04 & 25\% & 1.0e-08 & 4.7e+03 & 2.1e+01 & 1.2e-08 & 6.0e+01 & 2.7e+01 \\ 
  1e+04 & 50\% & 1.1e-08 & 4.7e+03 & 2.1e+01 & 1.3e-08 & 7.0e+01 & 3.0e+01 \\ 
  1e+04 & 75\% & 1.1e-08 & 4.7e+03 & 2.1e+01 & 2.3e-08 & 8.0e+01 & 3.1e+01 \\ 
  1e+04 & 100\% & 1.3e-08 & 4.7e+03 & 2.1e+01 & 9.9e-01 & 1.3e+02 & 3.3e+01 \\ 
  1e+04 & avg & 1.1e-08 & 4.7e+03 & 2.1e+01 & 4.0e-02 & 7.5e+01 & 2.9e+01 \\ 
  1e+04 & std & 7.0e-10 & 2.6e-10 & 5.3e-02 & 2.0e-01 & 2.3e+01 & 4.2e+00 \\ 
  1e+05 & 0\% & 1.0e-08 & 4.7e+03 & 2.1e+01 & 1.0e-08 & 4.2e+01 & 8.5e+00 \\ 
  1e+05 & 25\% & 1.0e-08 & 4.7e+03 & 2.1e+01 & 1.2e-08 & 5.3e+01 & 2.2e+01 \\ 
  1e+05 & 50\% & 1.1e-08 & 4.7e+03 & 2.1e+01 & 1.3e-08 & 5.6e+01 & 2.5e+01 \\ 
  1e+05 & 75\% & 1.1e-08 & 4.7e+03 & 2.1e+01 & 2.3e-08 & 6.4e+01 & 2.7e+01 \\ 
  1e+05 & 100\% & 1.3e-08 & 4.7e+03 & 2.1e+01 & 9.9e-01 & 7.8e+01 & 2.9e+01 \\ 
  1e+05 & avg & 1.1e-08 & 4.7e+03 & 2.1e+01 & 4.0e-02 & 5.8e+01 & 2.4e+01 \\ 
  1e+05 & std & 7.0e-10 & 4.5e-13 & 4.8e-02 & 2.0e-01 & 9.7e+00 & 4.6e+00 \\ 
  5e+05 & 0\% & 1.0e-08 & 4.7e+03 & 2.1e+01 & 1.0e-08 & 4.2e+01 & 8.5e+00 \\ 
  5e+05 & 25\% & 1.0e-08 & 4.7e+03 & 2.1e+01 & 1.2e-08 & 5.1e+01 & 2.2e+01 \\ 
  5e+05 & 50\% & 1.1e-08 & 4.7e+03 & 2.1e+01 & 1.3e-08 & 5.6e+01 & 2.5e+01 \\ 
  5e+05 & 75\% & 1.1e-08 & 4.7e+03 & 2.1e+01 & 2.3e-08 & 6.4e+01 & 2.7e+01 \\ 
  5e+05 & 100\% & 1.3e-08 & 4.7e+03 & 2.1e+01 & 2.6e-04 & 7.8e+01 & 2.8e+01 \\ 
  5e+05 & avg & 1.1e-08 & 4.7e+03 & 2.1e+01 & 1.0e-05 & 5.8e+01 & 2.4e+01 \\ 
  5e+05 & std & 7.0e-10 & 1.9e-13 & 3.5e-02 & 5.1e-05 & 9.9e+00 & 4.4e+00 \\ 
   \hline
\hline
\end{longtable}
\newpage
Results in 30 dimensions for functions 19, 20, 21, 22, 23, 24
% latex table generated in R 2.15.1 by xtable 1.7-0 package
% Sat Dec 15 19:01:56 2012
\begin{longtable}{llllllll}
  \hline
fe & stats & V19 & V20 & V21 & V22 & V23 & V24 \\ 
  \hline
1e+03 & 0\% & 8.7e+02 & 9.2e+02 & 5e+02 & NA & NA & NA \\ 
  1e+03 & 25\% & 9.3e+02 & 9.3e+02 & 5e+02 & NA & NA & NA \\ 
  1e+03 & 50\% & 9.3e+02 & 9.3e+02 & 5e+02 & NA & NA & NA \\ 
  1e+03 & 75\% & 9.3e+02 & 9.4e+02 & 5e+02 & NA & NA & NA \\ 
  1e+03 & 100\% & 9.4e+02 & 9.4e+02 & 5e+02 & NA & NA & NA \\ 
  1e+03 & avg & 9.3e+02 & 9.3e+02 &    NA & NA & NA & NA \\ 
  1e+03 & std & 1.4e+01 & 4.3e+00 &    NA & NA & NA & NA \\ 
  1e+04 & 0\% & 8.0e+02 & 8.0e+02 & 5e+02 & NA & NA & NA \\ 
  1e+04 & 25\% & 9.1e+02 & 9.1e+02 & 5e+02 & NA & NA & NA \\ 
  1e+04 & 50\% & 9.1e+02 & 9.1e+02 & 5e+02 & NA & NA & NA \\ 
  1e+04 & 75\% & 9.1e+02 & 9.1e+02 & 5e+02 & NA & NA & NA \\ 
  1e+04 & 100\% & 9.1e+02 & 9.1e+02 & 5e+02 & NA & NA & NA \\ 
  1e+04 & avg & 9.0e+02 & 9.0e+02 &    NA & NA & NA & NA \\ 
  1e+04 & std & 3.0e+01 & 2.1e+01 &    NA & NA & NA & NA \\ 
  1e+05 & 0\% & 8.0e+02 & 8.0e+02 & 5e+02 & NA & NA & NA \\ 
  1e+05 & 25\% & 9.0e+02 & 9.0e+02 & 5e+02 & NA & NA & NA \\ 
  1e+05 & 50\% & 9.0e+02 & 9.0e+02 & 5e+02 & NA & NA & NA \\ 
  1e+05 & 75\% & 9.0e+02 & 9.0e+02 & 5e+02 & NA & NA & NA \\ 
  1e+05 & 100\% & 9.1e+02 & 9.1e+02 & 5e+02 & NA & NA & NA \\ 
  1e+05 & avg & 9.0e+02 & 9.0e+02 &    NA & NA & NA & NA \\ 
  1e+05 & std & 2.9e+01 & 2.1e+01 &    NA & NA & NA & NA \\ 
  5e+05 & 0\% & 8.0e+02 & 8.0e+02 & 5e+02 & NA & NA & NA \\ 
  5e+05 & 25\% & 9.0e+02 & 9.0e+02 & 5e+02 & NA & NA & NA \\ 
  5e+05 & 50\% & 9.0e+02 & 9.0e+02 & 5e+02 & NA & NA & NA \\ 
  5e+05 & 75\% & 9.0e+02 & 9.0e+02 & 5e+02 & NA & NA & NA \\ 
  5e+05 & 100\% & 9.1e+02 & 9.1e+02 & 5e+02 & NA & NA & NA \\ 
  5e+05 & avg & 9.0e+02 & 9.0e+02 &    NA & NA & NA & NA \\ 
  5e+05 & std & 2.9e+01 & 2.1e+01 &    NA & NA & NA & NA \\ 
   \hline
\hline
\end{longtable}
\newpage

}
\section{Wnioski}
\label{sec-5}


Wyniki uzyskane w czasie tego projektu różnią się w porównaniu do
wyników uzyskanych przez J. Rönkkönen, S. Kukkonen, K. V. Price na
korzyść biblioteki DEoptim. Wyniki zarówno dla 10$^3$ jak i dla 5*10$^5$
wywołań funkcji są zazwyczaj lepsze o jeden do trzech rzędów
wielkości. Zaś warto zauważyć, że standardowe odchylenie uzyskanych
przez nas wyników jest zazwyczaj większe niż u wyników Rönkkönen et
al. Z tego można wywnioskować, że implementacja algorytmu
optymalizacji poprzez różniczkowanie ewolucyjne w postaci biblioteki
DEoptim dla języka R oraz dobrane przez nas parametry optymalizacji
potrafią uzyskać lepsze wyniki niż uzyskane przez Rönkkönen et al,
choć jej wyniki potrafią być mniej stabilne.

\end{document}