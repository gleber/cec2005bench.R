% Created 2012-12-09 Sun 20:07
\documentclass[11pt]{article}
\usepackage[utf8]{inputenc}
\usepackage[T1]{fontenc}
\usepackage{fixltx2e}
\usepackage{graphicx}
\usepackage{longtable}
\usepackage{float}
\usepackage{wrapfig}
\usepackage{soul}
\usepackage{textcomp}
\usepackage{marvosym}
\usepackage{wasysym}
\usepackage{latexsym}
\usepackage{amssymb}
\usepackage{hyperref}
\tolerance=1000
\providecommand{\alert}[1]{\textbf{#1}}

\title{Wstęp do algorytmów ewolucyjnych}
\author{Gleb Peregud}
\date{\today}
\hypersetup{
  pdfkeywords={},
  pdfsubject={},
  pdfcreator={Emacs Org-mode version 7.8.11}}

\begin{document}

\maketitle

\setcounter{tocdepth}{3}
\tableofcontents
\vspace*{1cm}

Dokumentacja końcowa

\section{Wstęp}
\label{sec-1}


Projekt polega na zaimplementowaniu benchmark CEC2005 w języku R,
użyciu pakietu DEoptim do zbadania efektywności optymizacji
Differential Evolution na tych funkcjach oraz porównaniu ich z
wynikami uzyskanymi przez J. Rönkkönen, S. Kukkonen oraz K. V. Price.

Dany projekt skupia się na zbadaniu następujących funkcji z benchmarku
CEC2005:
\begin{itemize}
\item Shifted Rosenbrock’s Function (funkcja numer 6)
\item Shifted Rotated Griewank’s Function without Bounds (funkcja numer 7)
\item Shifted Rotated Ackley’s Function with Global Optimum on Bounds (funkcja numer 8)
\item Shifted Rastrigin’s Function (funkcja numer 9)
\item Shifted Rotated Rastrigin’s Function (funkcja numer 10)
\item Shifted Rotated Weierstrass Function (funkcja numer 11)
\item Schwefel’s Problem 2.13 (funkcja numer 12)
\item Rotated Hybrid Composition Function with a Narrow Basin for the
  Global Optimum (funkcja numer 19)
\item Rotated Hybrid Composition Function with Global Optimum on the
  Bounds (funkcja numer 20)
\item Rotated Hybrid Composition Function (funkcja numer 21)
\item Rotated Hybrid Composition Function with High Condition Number
  Matrix (funkcja numer 22)
\item Non-Continuous Rotated Hybrid Composition Function (funkcja numer 23)
\item Rotated Hybrid Composition Function (funkcja numer 24)
\end{itemize}

Funkcje zostały zbadane dla 10 oraz 30 wymiarów.
\section{Algorytm}
\label{sec-2}


DEoptim jest implementacja >
Wybrane parametry >
\section{Implementacja}
\label{sec-3}


Żeby nie powtarzać pracy wykonanej przez innych w projekcie został
wykorzystany projekt cec2005benchmark w języku R autorstwa Yasser
González Fernández <ygonzalezfernandez@gmail.com> oraz Marta Rosa Soto
Ortiz <mrosa@icimaf.cu>.

Implementacja użyta do uzyskania wyników jest dostępna na stronie
\href{https://github.com/gleber/cec2005bench.Rpr}{https://github.com/gleber/cec2005bench.Rpr}

Ważnym punktem implementacji była możliwość uruchamiania procesu
optymizacji po jej przerwaniu, żeby móc efektywnie wprowadzać jakieś
poprawki do systemu bez tracenia wyników jej pracy. Zostało to
zaimplementowane korzystając z możliwości serializacji macierzy i
zapisywaniu ich do pliku korzystając z funkcji 

Kod - opis kodu, wykorzystanych technologii oraz optymalizacji
\section{Wyniki}
\label{sec-4}


Tabelki z porównaniem wyników uzyskanych przez J. Rönkkönen, S. Kukkonen, K. V. Price
\section{Wnioski}
\label{sec-5}


Słowne podsumowanie wyników i jakieś wnioski

\end{document}